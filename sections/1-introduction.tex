\section{Einleitung}
\label{section:introduction}

Die Clusteranalyse beschäftigt sich mit Verfahren, die Ähnlichkeitsstrukturen in (großen) Datenmengen identifizieren. Der englische
Begriff \emph{cluster} lässt sich in diesem Kontext am ehesten mit Gruppe oder Ansammlung übersetzen. Dem Gebiet liegt die
Annahme zu Grunde, dass in gegebenen Datenmengen typischerweise Strukturen existieren, die mit einem geeigneten Ähnlichkeitsmaß
erkannt werden können. Diese grundlegende Annahme ist insofern plausibel, als dass die zu analysierenden Datenmengen in der Praxis
in der Regel aus einem konkreten Anwendungsfall gesammelt oder aufgezeichnet werden, in dem bestimmte Muster oder sich wiederholende
Prozesse existieren. Bei der Analyse von Daten aus natürlichen Prozessen, sind diese beispielsweise oft gemäß einer bestimmten
Wahrscheinlichkeitsverteilung verteilt. Hinzu kommt, dass Daten ohne jegliche Strukturen zumeist ohnehin nicht besonders
aufschlussreich sind.
Sind die vorhandenen Strukturen einmal erkannt, kann der gesamte Datensatz in Gruppen oder \emph{Cluster}
gegliedert werden, sodass innerhalb eines Clusters große Ähnlichkeit zwischen Objekten besteht, während Objekte, die
unterschiedlichen Clustern zugeordnet sind, geringe Ähnlichkeit aufweisen. Anhand der Cluster, die mit Verfahren der Clusteranalyse
gebildet wurden, lassen sich dann häufig aufschlussreiche Aussagen über die Eingabedaten treffen, die beim bloßen Anblick der
gesamten unstrukturierten Datenmenge nicht möglich gewesen wären. Das gilt insbesondere, wenn die Daten hochdimensional sind,
sodass eine graphische Auswertung auch nicht möglich ist.
\absatz
Wir betrachten dazu einleitend ein einfaches Beispiel der Gegenwart: die Analyse von Strukturen in sozialen Netzwerken.
Die Eingabeobjektmenge besteht in diesem Fall aus Objekten, die Informationen über Beziehungen und Aktivitäten zwischen
Nutzern des sozialen Netzwerks enthalten. Mit Hilfe der Clusteranalyse können beispielsweise Gruppen von Benutzern
identifiziert werden, die viel miteinander interagieren und somit eine soziale Community, also eine Gruppe von Freunden
oder Bekannten, bilden. In diesem Beispiel kann es auch aufschlussreich sein, sogenannte "`schwarze Löcher"' zu identifizieren,
also Benutzer, die eine große Menge von eingehenden Aktivitäten zahlreicher anderer Benutzer haben, selbst aber nur selten
mit einzelnen Benutzern interagieren. Bei Benutzern mit diesem Verhalten handelt es sich häufig um Prominente mit einer großen
Fangemeinde. Unabhängig vom konkreten verfolgten Ziel wird schnell
klar, dass entsprechende Aussagen gerade bei Netzwerken mit einer großen Nutzeranzahl, wie beispielsweise Facebook, nur dann
möglich sind, wenn Verfahren der Clusteranalyse eingesetzt werden. Würde man lediglich die unstrukturierte gesamte Datenmenge
betrachten, ließen sich nur sehr schwer Informationen gewinnen, die keine Aussagen über das globale Netzwerk treffen. Es wird
ebenso klar, dass es nötig ist, Verfahren zu entwickeln, die innerhalb der Grenzen gegenwärtiger Hardware Ergebnisse liefern, wenn
die Eingabedatenmenge sehr große bis gigantische Größenordnungen annimmt.
\absatz
In dieser Arbeit beschäftigen wir uns unter anderem mit einem weit verbreiteten Clusteringmodell, in dem das Ähnlichkeitsmaß die
(quadrierte) geometrische Distanz ist. Die Eingabe besteht demnach aus einer Menge $P$ von Punkten aus dem euklidischen Raum
$\Rd$. Neben der Punktmenge $P$ enthält unsere Eingabe zusätzlich eine positive ganze Zahl $k$, welche die Anzahl an Clustern
angibt, in welche die Punkte unterteilt werden sollen. Dieses Modell ist unter dem Namen "`$k$-means-Problem"' bekannt. Den
Algorithmen für das $k$-means-Problem, die wir betrachten wollen, ist gemein, dass sie eine initiale Auswahl
von Clustern und zugehörigen \emph{Zentren} iterativ weiter verbessern, bis eine zumindest lokal optimale Lösung ermittelt wurde.
Die eigentliche Aufgabe besteht dann darin, die Summe der
(quadrierten) Distanzen der Punkte eines Clusters zu ihrem Clusterzentrum zu minimieren.

Da das $k$-means-Problem schon für $k = 2$ NP-hart ist~\cite{AloiseDHP09}, sind für praktische Zwecke insbesondere
Heuristiken, also Algorithmen ohne feste Schranken für Laufzeit oder Qualität der Lösung) und
Approximationsalgorithmen, die eine nicht-optimale Lösung berechnen, welche jedoch garantiert nur um einen gewissen
Faktor von der optimalen Lösung abweichen, interessant. Wir betrachten die Konstruktion von \emph{Kernmengen}.
Bei Eingabe einer Punktmenge $P$ für das $k$-means-Problem handelt es sich bei einer Kernmenge um eine kleinere gewichtete
Punktmenge $S$ mit der Eigenschaft, dass für \emph{jede} Wahl von $k$ Clusterzentren die Summe der quadrierten Distanzen der Punkte in
$P$ zu den Zentren in etwa so groß ist wie die Summe der quadrierten Distanzen der Punkte in $S$ zu den Zentren. Effektiv ist
eine Kernmenge also ein hinsichtlich der Clusteringzielfunktion repräsentatives, (teilweise deutlich) kleineres Abbild der
ursprünglichen Punktmenge. Üblicherweise bieten Kernmengenkonstruktionen Garantien einer $(1 + \epsilon)$-Approximation,
das heißt, dass ein $k$-means-Clustering auf der Kernmenge höchstens um einen Faktor $(1 + \epsilon)$ schlechter ist, als das
selbe Clustering auf der Ursprungsmenge $P$. Neben der Approximation des $k$-means-Problems können Kernmengen auch ein
effektives Mittel im Umgang mit großen Datenmengen sein. Wenn die Größe der Kernmenge $S$ nur ein kleiner Anteil an der Größe
von $P$ ist, können Clustering-Algorithmen auf $S$ deutlich schneller ausgeführt werden als auf $P$.
\absatz
Schließlich betrachten wir in dieser Arbeit noch eine weitere Modellierung von Clusteringproblemen, die mit der geometrischen
Auffassung verwandt ist. Für einige Anwendungen ist es sinnvoll, die zu clusternden Objekte als Knoten eines gewichteten
Graphen zu betrachten. Das Ähnlichkeitsmaß ist dann nicht mehr die euklidische Distanz, sondern das Gewicht der Kanten zwischen
den Objekten beziehungsweise Knoten. Stellt man sich den gezeichneten Graphen vor, kommt das Kantengewicht der "`Distanz"'
zwischen den Knoten gleich. Bei der Clusteranalyse von Graphen können wir unter anderem auf diverse Ergebnisse aus der
Graphentheorie zurückgreifen und zudem Techniken aus der linearen Algebra anwenden, wie wir später aufzeigen werden.

\paragraph{Anwendungen.} Die Clusteranalyse war in der Vergangenheit bereits Gegenstand intensiver Forschung, die bis heute
regelmäßig neue theoretische Erkenntnisse liefert. Es handelt sich aber um ein Gebiet mit hoher praktischer Relevanz, da
Clustering in zahlreichen Anwendungen entweder selbst durchgeführt werden muss, oder als Teilproblem in einem übergeordneten
Kontext auftritt.
Wir wollen an dieser Stelle einen kurzen und sicherlich unvollständigen Überblick über einige wichtige Anwendungen
der bisher vorgestellten Clusteringmodelle geben.

Eine gegenwärtige Anwendung für $k$-means-Clustering haben wir mit der Analyse sozialer Netzwerke bereits erwähnt. Eine weitere
Anwendung, die gerade heute im Bereich der Online-Shops relevant ist, findet das $k$-means-Clustering in der Marktforschung.
Hier sollen Informationen über das Einkaufsverhalten von Kunden genutzt werden, um diese beispielsweise den Marktsegmenten nach
zu unterteilen. Je nachdem, wie detailliert die gesammelten Informationen sind, können diese auch genutzt werden, um
Produktplatzierung zu optimieren, die Nachfrage nach neuen Produkten abzuschätzen, oder potenzielle neue Märkte zu
erkennen. Neben den wirtschafts- und sozialwissenschaftlichen Anwendungsfällen, kommt die Clusteranalyse auch in
geographischen und geologischen Zusammenhängen zum Einsatz. Konkret werden hier zum Beispiel chemische Eigenschaften von gesammelten
Proben geclustert, um geographische Profile der untersuchten Substanzen zu erstellen.
\absatz
Kernmengen werden in drei Szenarien eingesetzt. Wie bereits erwähnt bieten viele Kernmengenkonstruktionen
eine garantierte Approximationsgüte, das heißt das anhand der Kernmenge berechnete Clustering weicht für jede beliebige
Wahl der Zentren aus $\Rd$ um beispielsweise einen Faktor von höchstens $1 + \epsilon$ von einem Clustering mit den
selben Zentren auf der Ursprungsmenge ab. Man spricht in diesem Zusammenhang auch von \emph{starken} Kernmengen.
Ist diese Eigenschaft erfüllt, können wir mit einer Kernmenge, die eine deutlich kleinere Größe hat als die Ursprungsmenge,
für Approximationsalgorithmen für das $k$-means-Problem einen signifikanten \emph{Speedup}, also eine Beschleunigung der
Ausführungszeiten, erzielen. Dabei muss selbstverständlich beachtet werden, dass die Konstruktion der Kernmenge selbst
ebenfalls Zeit in Anspruch nimmt und nicht zu aufwändig sein darf, wenn ein Speedup beabsichtigt wird. 

Ein relativ neues Modell für Algorithmen sind sogenannte \emph{Datenströme}, bei denen die Eingabe nicht initial vollständig
vorliegt, sondern kontinuierlich in einem Datenstrom geliefert wird. Algorithmen für Datenströme, also
\emph{Datenstromalgorithmen} dürfen die Daten des Stroms nur genau einmal einlesen und sind zusätzlich polylogarithmisch
bezüglich der Eingabegröße platzbeschränkt, das heißt, es ist nicht möglich, die gesamte Eingabe zu protokollieren.
Beim Clustering in Datenströmen nutzt man aus, dass die Vereinigung einer $(1 + \epsilon_1)$-Kernmenge und einer
$(1 + \epsilon_2)$-Kernmenge eine $(1 + \epsilon_1) (1 + \epsilon_2)$-Kernmenge ergibt. Die Eingabe wird in kleinere Teile
partitioniert, für die jeweils eine Kernmenge berechnet wird. Die Kernmengen werden dann zu einer einzelnen Kernmenge
vereinigt. Da jede Vereinigung einen zusätzlichen Fehler verursacht, darf dies nicht zu oft erfolgen. Man setzt dazu die
Merge-and-Reduce-Technik ein, die in einer Baum-Struktur bottom-up die Kernmengen einer Partition miteinander vereinigt.
Mit der Technik wird insbesondere sichergestellt, dass jeder einzelne Punkt der Eingabe nur in $\log \left|P\right|$ vielen
Reduktionsschritten verwendet wird, wobei $\left|P\right|$ die Anzahl an bisher betrachteten Eingabepunkten ist.
Die Anwendung von Kernmengen im Bereich von
Datenstromalgorithmen ist Gegenstand gegenwärtiger Forschung und hat bereits erste Ergebnisse erbracht, wie wir später noch
sehen werden.

Schließlich können wir Kernmengen mit einer ähnlichen Idee einsetzen, um Clusteringalgorithmen parallel verteilt auszuführen.
Wenn die Eingabepunktmenge verteilt vorliegt, berechnen wir auf den Teilen verteilt oder parallel Kernmengen, übermitteln die
berechneten Kernmengen an eine zentrale Instanz und vereinen diese dort zu einer gesamten Kernmenge. Dadurch vermeiden wir es,
die gesamte Punktmenge über das Netzwerk übermitteln zu müssen.
\absatz
Die Clusteranalyse von Graphen findet ihre vielleicht wichtigste Anwendung in der Bildsegmentierung. Sie wird in diesem
Kontext eingesetzt, um bei digitalen Bildern Kanten und insbesondere Objekte zu erkennen. Die Kantenerkennung ist eine
fundamentale Technik in der digitalen Bildverarbeitung, die als Vorverarbeitung für speziellere Verfahren nötig ist. Das Erkennen
von Objekten macht beispielsweise die Erkennung von Gesichtern, Personen oder Orten auf Fotos möglich.

Darüber hinaus ergibt sich eine natürliche Anwendung in der Identifikation relevanter Strukturen und insbesondere der
Konnektivitätsanalyse in Netzwerken. Auch die Entdeckung von häufigen Anrufmustern in der Telekommunikation kann hier
von Interesse sein. Schließlich kann es in Netzwerken mit dynamischer Topologie, wie beispielsweise \emph{ad-hoc-Netzwerken},
die in intelligenten Umgebungen mit vielen wechselnden mobilen Geräten vorkommen, hilfreich sein, zu clustern. Lokales Clustering
wird hier unter anderem eingesetzt, um Routing-Algorithmen zu verbessern. Das Netzwerk wird dabei als Graph aufgefasst.

Zudem kann die Clusteranalyse von Graphen eingesetzt werden, um Caching-Mechanismen in Datenbanken zu optimieren.
Konkret werden in Datenbanksystemen die einzelnen Einträge in Form von \emph{Seiten} auf dem Speichermedium abgelegt,
von denen einige im Hauptspeicher gecached werden. Ein gutes Clustering führt in diesem Kontext dazu, dass das Laden eines
Clusters, dem ein angefragtes Datum zugeordnet ist, zudem relevante Daten liefert, die für ähnliche zukünftige Anfragen
erforderlich sind.

\paragraph{Verwandte Arbeiten.} Der Algorithmus von Lloyd~\cite{Lloyd82} ist eine schon früh entwickelte Heuristik für das
$k$-means-Problem, die bis heute von praktischer Relevanz ist, da sie im Allgemeinen gute Clusterings produziert, obwohl
der Algorithmus keine Approximationsgüte garantiert. Wir betrachten in dieser Arbeit insbesondere die Verbesserung von
Lloyds Algorithmus durch Arthur und Vassilvitskii~\citep{ArthurV07}, die eine $\BigO{\log k}$-Approximation für das
$k$-means-Problem ist.

Das heutige Konzept von \emph{starken} Kernmengen wurde in~\cite{HarPeledM04} vorgestellt. Wir beschäftigen uns in dieser Arbeit
hauptsächlich mit der Konstruktion aus~\cite{FeldmanSS13,Schmidt14}.

Das Clusteringmodell in Graphen kann bis zu~\cite{KernighanL70} zurückverfolgt werden. Die erste Anwendung von sogenannten
\emph{spektralen} Methoden für die Clusteranalyse von Graphen und insbesondere die Anwendung im Bereich der Bildsegmentierung
stammt von~\cite{ShiM00}. Eine wichtige Weiterentwicklung des Ansatzes geht zurück auf~\cite{NgJW01}. Die unmittelbare
Äquivalenz zum geometrischen $k$-means-Problem wurde in~\cite{DhillonGK04,DhillonGK07} gezeigt. Wir werden sehen, dass diese
für unsere Zwecke besonders wichtig ist.
\absatz
Die Arbeit ist folgendermaßen gegliedert. In Kapitel~\ref{section:basics} definieren wir die wichtigsten Begriffe und
Konzepte, die wir im Rahmen dieser Arbeit benötigen. Außerdem stellen wir die Algorithmen vor, die wir als Grundlage verwenden,
oder weiterentwickeln beziehungsweise verbessern wollen. In Kapitel~\ref{section:main} entwickeln wir zum einen eine
Verbesserung des Algorithmus von Dhillon, Guan und Kulis~\cite{DhillonGK04,DhillonGK07} und zeigen zum anderen, mit welchen
Modifikationen die Kernmengenkonstruktion von Feldman, Sohler und Schmidt~\cite{FeldmanSS13,Schmidt14} praktisch umsetzbar wird.
Die entwickelten Algorithmen wollen wir dann in Kapitel~\ref{section:experiments} anhand einer Reihe von Experimenten empirisch
bewerten. Unsere Ergebnisse fassen wir schließlich in Kapitel~\ref{section:conclusion} zusammen.