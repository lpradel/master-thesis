\section{Zusammenfassung}
\label{section:conclusion}

Die Kernel Methode ist eine wirkungsvolle Technik in der Clusteranalyse. Ihr Einsatz erlaubt es, die Effektivität von Verfahren
signifikant zu steigern, die auf die lineare Separierung von Clustern beschränkt sind. Angewandt auf den $k$-means-Algorithmus
erhält man den Kernel-$k$-means-Algorithmus, mit dem sich beispielsweise das Graph\-partitionierungsproblem lösen lässt, wie wir
in Abschnitt~\ref{subsection:wkkm-graphcut-graphclustering} erläutert haben.
\absatz
Der \kmpp-Algorithmus sieht eine geschicktere Wahl der initialen Zentren für den $k$-means-Algorithmus vor.
Wir haben die Kernel Methode auf \kmpp{} angewandt und den resultierenden \kkmpp-Algorithmus vorgestellt.
Dabei konnten wir zeigen, dass die $\mathcal{O}(\log k)$-Approximationseigenschaft von \kmpp{} erhalten bleibt.
Bislang haben in der Graphpartitionierung spektrale Verfahren die qualitativ besten Ergebnisse geliefert.
Diese bringen jedoch wegen der nötigen Eigenvektorberechnungen hohe Laufzeiten mit sich.
Mit \kkmpp{} war es uns möglich, Graphpartitionierungen ohne spektrale Techniken zu berechnen, die qualitativ kompetitiv zu
den bisherigen Verfahren sind. Wir konnten jedoch gleichzeitig die Laufzeiten um einen Faktor zwei verbessern. Dies ist besonders
erfreulich, wenn man bedenkt, dass das modifizierte Framework Graclus die derzeit vielleicht schnellste quelloffene
Software im Bereich der Graphpartitionierung ist, die von einer Reihe von quelloffenen Bildverarbeitungsprogrammen eingesetzt wird.
\absatz
Wir haben zudem eine Variante der Kernmengenkonstruktion von Feldman, Schmidt und Sohler~\cite{FeldmanSS13} vorgestellt,
die in den Experimenten praktikable Qualität und Laufzeit aufweist. Insbesondere die Kernel-basierte Variante liefert bessere
Ergebnisse als der \kmpp-Algorithmus. Mit der Kernmengenkonstruktion war es uns möglich, den Kernel-$k$-means-basierten 
Algorithmus zur Graphpartitionierung noch einmal zu verbessern.
\absatz
Über diese Arbeit hinaus wäre es sicherlich interessant, zu untersuchen, ob die Kernel Methode auch auf andere $k$-means-Algorithmen
angewandt werden kann und ob dies zu Verbesserungen führt.

Für die Kernmengenkonstruktion haben wir in der Einleitung eine mögliche Anwendung in der verteilten Clusteranalyse erwähnt.
Es wäre daher vermutlich aufschlussreich, eine verteilte Variante der Konstruktion zu implementieren und diese dann auf verteilt
vorliegenden zu clusternden Daten zu untersuchen. Dabei sollte der Algorithmus nicht nur auf einigen wenigen Maschinen ausgeführt
werden, sondern auf 16 oder mehr Knoten, um die Leistungsfähigkeit in verteilten Systemen zu analysieren.