\section{Grundlegende Definitionen und Algorithmen}
\label{section:basics}

In diesem Kapitel definieren wir die für unsere Zwecke relevanten Begriffe im Kontext der Clusteranalyse und führen die
populärsten grundlegenden Algorithmen ein, deren Ideen für uns im Folgenden noch von Bedeutung sein werden. Wir gehen dabei
nach Themengebieten geordnet vor: In Abschnitt~\ref{subsection:basics:clustering} skizzieren wir kurz das
Themengebiet der Clusteranalyse, definieren die wichtigsten Zielfunktionen und stellen zwei wichtige Algorithmen vor.
Abschnitt~\ref{subsection:basics:graphs} führt kurz in die Graphentheorie sowie die Clusteranalyse von Graphen ein. In diesem
Abschnitt werden wir zudem die von der klassischen Clusteranalyse sehr unterschiedlichen Optimierungskritierien für die
Clusteranalyse von Graphen herausstellen. Schließlich fassen wir in Abschnitt~\ref{subsection:basics:kernel-spectral} die
wichtigsten Methoden und Algorithmen aus dem Bereich der spektralen Clusteranalyse zusammen und stellen zudem die wichtigsten
Konzepte von Kernel-Methoden vor.

\subsection{Clustering und \texorpdfstring{$k$}{k}-means}
\label{subsection:basics:clustering}

Clusteranalyse oder "`Clustering"' beschäftigt sich mit der Einteilung von Objekten in Gruppen ("`Cluster"'), sodass
sich die Objekte innerhalb eines Clusters gemäß eines bestimmten Optimierungskriteriums ähnlich sind und von Objekten eines
anderen Clusters unterscheiden. Es existieren zahlreiche grundsätzlich verschiedene Ansätze, Clusteringprobleme zu lösen.
Wir beschränken uns in dieser Arbeit auf \emph{partitionierende} Clusteringprobleme und -verfahren. Bei diesen
soll eine Menge von $d$-dimensionalen Punkten, welche der erste Teil der Eingabe ist, gemäß einer Cluster-Zielfunktion möglichst
optimal in genau $k$ Cluster unterteilt werden, wobei $k$ der ganzzahlige zweite Teil der Eingabe ist.

Für die Zielfunktion, welche die Nähe oder Ferne von Punkten zueinander quantifiziert, sind bei Eingabepunkten aus
$\Rd$ Metriken naheliegend. Intuitiv ist dabei die euklidische Distanz, welche als Zielfunktion für die beiden bekanntesten
Clustering-Problemstellungen dient.

\begin{definition}[$k$-median und $k$-means]
Sei $P \subset \Rd$ und $k \in \mathbb{N}^{+}$. Das $k$-median-Problem besteht darin, eine Menge von $k$ (Cluster-)\emph{Zentren}
$C = \{ c_1, \dots, c_k \}$ mit $c_i \in \Rd$ zu finden, sodass der folgende Term minimal wird:
\[ \sum_{p \in P} \min_{c_i \in C} \Euclid{p}{c} \]
Das $k$-means-Problem unterscheidet sich nur darin, dass bei diesem die Summe der \emph{quadrierten} euklidischen Distanzen
zum jeweils nächstgelegenen Zentrum minimiert werden soll, das heißt, dass der folgende Term minimiert werden soll:
\[ \sum_{p \in P} \min_{c_i \in C} \EuclidSquared{p}{c} \]
Beim \emph{gewichteten} $k$-means-Problem werden den Eingabepunkten zusätzlich mit einer Funktion $w : P \rightarrow \mathbb{R}$
Gewichte zugewiesen. Die zu minimierende Zielfunktion lautet dann entsprechend
\[ \sum_{p \in P} \min_{c_i \in C} w(p) \EuclidSquared{p}{c} \]
\end{definition}
Sowohl das $k$-Median-Problem~\cite{MegiddoS84} als auch das $k$-means-Problem~\cite{AloiseDHP09} sind optimal NP-schwer lösbar.
Typischerweise werden zur Lösung daher approximative oder heuristische Algorithmen eingesetzt.

\subsection{Graphen und Clusteranalyse von Graphen}
\label{subsection:basics:graphs}

\subsection{Kernel-Methoden und spektrales Clustering}
\label{subsection:basics:kernel-spectral}